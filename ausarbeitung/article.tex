\documentclass{hhuarticle}
% \usepackage[ngerman,english]{babel} % English
\usepackage[english,ngerman]{babel} % Deutsch

\title{Gestalten der schriftlichen Ausarbeitung}

% Vorlange von Jannik Dunkelau und Joshua Schmid
\author{John Witulski} % keine Matrikelnummer eintragen

\subject{Bachelorseminar: Programmiersprachen} % Optional.
\semester{Wintersemester 2022}

% \printblackwhitetrue % Vereinfacht Schwarzweißdruck (Logo, Plots, ...).

%%%%%%%%%%%%%%%%%%%%%%%%%%%%%%%%%%%%%%%%%%%%%%%%%%%%%%%%%%%%%%%%%%%%%%%%%%%%%%%%
%% LaTeX Packages in Nutzung                                                  %%
%%                                                                            %%
%% Im folgenden können Sie für die Niederschrift benötigte LaTeX-Pakete       %%
%% einbinden.                                                                 %%
%% Diese Vorlage kommt bereits mit einigen nützlichen inkludierten Paketen.   %%
%%%%%%%%%%%%%%%%%%%%%%%%%%%%%%%%%%%%%%%%%%%%%%%%%%%%%%%%%%%%%%%%%%%%%%%%%%%%%%%%

%% Häufig benutzte mathematische Packages.
\usepackage{amsfonts}
\usepackage{amsmath}
\usepackage{amssymb}

\usepackage{enumitem} % Leichter konfigurierbare enumerate-Umgebungen.
\usepackage{subcaption} % Unterteilung von Figures in Subfigures.
\usepackage[colorlinks]{hyperref} % Klickbare Links (z.B. Inhaltsverzeichnis).
\sethyperrefhhucolors{} % Setzt den Farbsatz der HHU für hyperref.
\usepackage{url} % \url Kommando für Darstellung von Links
\usepackage{csquotes} % Improved quoting.

%% Tabellen
\usepackage{tabularx} % tabularx Umgebung für mehr Kontrolle über Tabellen.
\usepackage{booktabs} % \toprule, \midrule, \bottomrule
\usepackage{multirow}
\usepackage{multicol}
\usepackage{longtable} % Große Tabellen gehen über mehrere Seiten.

%% Quellcode
\usepackage{listings} % Einbindung von Code.
\setlstlistingstyle{} % Kosmetische Einstellungen
% Sprachabhängige Bezeichnung.
\iflanguage{ngerman}{\renewcommand{\lstlistingname}{Quellcode}}{}

%% Algorithmen in Pseudocode
\usepackage{algorithm} % Float-Umgebung für angegebene Algorithmen.
\usepackage{algorithmicx} % Angabe von Algorithmen in Pseudocode.
\usepackage{algpseudocode} % Standart Pseudocode-Elemente für Algorithmen.
\algsmallfont{}
% Sprachabhängige Bezeichnung.
\iflanguage{ngerman}{\floatname{algorithm}{Algorithmus}}{}

%% Intelligenteres Referenzieren mittels \cref.
\usepackage[capitalize,noabbrev,ngerman]{cleveref}
%\pagenumbering{gobble}

\definecolor{pblue}{rgb}{0.13,0.13,1}
\definecolor{pgreen}{rgb}{0,0.5,0}
\definecolor{pred}{rgb}{0.9,0,0}
\definecolor{pgrey}{rgb}{0.46,0.45,0.48}
\lstset{language=Java,
  showspaces=false,
  showtabs=false,
  breaklines=true,
  showstringspaces=false,
  breakatwhitespace=true,
  commentstyle=\color{pgreen},
  keywordstyle=\color{pblue},
  stringstyle=\color{pred},
  basicstyle=\ttfamily,
  moredelim=[il][\textcolor{pgrey}]{$$},
  moredelim=[is][\textcolor{pgrey}]{\%\%}{\%\%}
}

\begin{document}

  \maketitle

  \begin{abstract}
    Haskell ist eine Sprache, welche vor mehr als 30 Jahren zuerst erschien
    und dennoch heutzutage immer noch ihrer Zeit voraus ist.
    Trotz dessen ist sie in der Industrie wenig verbreitet, da nur
    wenige Programmierer mit ihr Erfahrung haben.
    Warum sich dies ändern sollte und warum nicht nur Haskell, sondern
    auch andere Sprachen, sowie Programmierer davon profitieren können,
    wird in dieser Arbeit erläutert.
  \end{abstract}

\tableofcontents


  \section{Einleitung}

  Im September 1987 wurde die
  ''Conference on Functional Programming Languages and Computer Architecture``
  in Portland, Oregon, USA, abgehalten. Ziel war es, die zersplitterte
  funktionale Programmiersprachen-Community zu vereinen und eine
  gemeinsame Standardsprache zu entwickeln.
  Die Sprache sollte alle Vorteile der bis dahin existierenden
  funktionalen Sprachen vereinen und gleichzeitig die Nachteile
  beseitigen.
  Die Sprache welche dabei entstand, ist Haskell.

  Während ihrer Entwicklung wurde Haskell stark von Akademikern beeinflusst,
  wobei insbesondere Themen wie Typtheorie, Kategorientheorie und das Lambda-Kalkül
  eine große Rolle spielten. Daraus resultierte, dass Haskell
  einige Besonderheiten aufweist, welche es von anderen Sprachen unterscheidet.
  Kurzgesagt ist Haskell eine polymorphe, funktionale, statisch-typisierte Programmiersprache mit
  lazy-evaluation, wobei diese Begriffe im Folgenden erläutert werden.

  Diese Eigenschaften machen Haskell zu einer sehr mächtigen Sprache, welche
  es ermöglicht, elegante und effiziente Lösungen für viele Probleme zu finden.
  Durch ihren starken mathematisch, theoretischen Grundaufbau besitzt Haskell
  Strukturen tief in sich verankert, welche noch heutzutage
  von anderen Programmiersprachen teilweise übernommen werden. Beispiele
  dafür sind algebraische Datentypen, welche nun auch in Rust vorkommen,
  oder die lazy-evaluation, welche in Java mittels Streams eingeführt wurde.

  Jede Ausarbeitung beginnt mit einer guten Einleitung.
  Hier liegt der Fokus auf den Grundlagen,
  sodass klar ist, worum es geht, was das Thema behandelt und
  wieso dies in erster Linie interessant ist.
  Detaillierte Informationen folgen in den späteren Abschnitten.

  Diese grundlegende Einleitung erlaubt auch,
  dass spätere Informationen hierauf aufbauen können.
  Dadurch lässt sich verhindern,
  dass zwei unterschiedliche Abschnitte die gleichen Grundlagen definieren
  und somit im Dokument doppeln.


  \section{Struktureller Aufbau}

  Es ist ein zusammenhängender Text von 15--20 Seiten (+ Referenzen) zu verfassen,
  der das gegebene Thema und die Lernziele abdeckt.
  Achten Sie hierbei auf eine sinnvolle Gliederung der Unterthemen.

  \subsection{Unterthemen}
  Unterthemen geben bereits eine Aufteilung in Abschnitte vor und jedes
  Unterthema sollte nicht mehr als einen Abschnitt umfassen.
  Dabei sollte nach dem Titel des Abschnitts je ein
  kurzer, einführender Text stehen, bevor der Abschnitt in weitere,
  sinnvolle Unterabschnitte aufgeteilt wird.
  Dieser einführende Text steht direkt hinter dem jeweiligen Titel und
  nicht in einem etwaigen Unterabschnitt.


  \section{Abbildungen, Tabellen und Beispiele}

  Sie werden in Ihren Zusammenfassungen nicht darum herum kommen,
  Abbildungen, Tabellen und Beispiele zur besseren Anschaulichkeit zu nutzen.
  Diese Abbildungen müssen im Text referenziert und besprochen werden
  und sollten nicht alleine stehen.
  Ebenfalls sollten Abbildungen und Tabellen zentriert dargestellt werden
  (mit \texttt{\textbackslash centering}).

  Mit \cref{fig:initial-draft} wird beispielhaft die Referenzierung einer Grafik
  demonstriert.
  Auf \LaTeX-Seite geschieht dies durch das Setzen eines Labels
  (\texttt{\textbackslash label}) für die Grafik
  und durch die Nutzung der Befehle
  \texttt{\textbackslash ref} oder \texttt{\textbackslash cref}.

  \begin{figure}[h]
    \centering
    \includegraphics[width=4cm]{fig/hhulogo.pdf}
    \caption{Das neue HHU-Logo.}%
    \label{fig:initial-draft}
  \end{figure}

  Der \texttt{\textbackslash cref}-Befehl entstammt hierbei dem
  \texttt{cleveref}-Paket und setzt automatisch den korrekten Bezeichner
  vor die Referenz, in diesem Beispiel ``Abbildung''.
  Zum Vergleich: \texttt{\textbackslash ref} setzt nur die
  Nummer \ref{fig:initial-draft}, die der Abbildung zugeteilt wurde.

  \subsection{Beispiele}
  Viele Themen profitieren von einem guten Beispiel,
  um Konzepte oder Eigenschaften klarer zu präsentieren.
  Achten Sie bei Beispielen darauf,
  dass diese nicht unnötig komplex sind und nur auf das nötige reduziert sind.
  Sollte es sich anbieten, ist es hilfreich ein Running Example zu nutzen,
  welches von allen Unterthemen aufgegriffen wird.
  So ist der Leser bereits mit den Gegebenheiten des Beispiels vertraut
  und kann Unterschiede zwischen den verschiedenen Szenarien genauer
  differenzieren.


  \subsection{Code}

  \begin{figure}[h]

\begin{lstlisting}
public class HelloWorld 
{
 
       public static void main (String[] args)
       {
             // Ausgabe Hello World!
             System.out.println("Hello World!");
       }
}
\end{lstlisting}
 \caption{So binden Sie Code ein}%
  \label{fig:codeBlock}
\end{figure}

  \subsection{Tabellen}

\begin{table}[h!]
\centering
\begin{tabular}{ |c|c|c| } 
 \hline
 Sprache & Jahr & Version \\ 
\hline
 Java & 1997 & JSE 19.0 \\ 
 C & 1972 & C17 \\ 
 \hline
\end{tabular}
\caption{Eine sinnfreie Tabelle}
\label{table:1}
\end{table}

  \section{Literaturverweise}

  Die Angabe von Quellen und Verweisliteratur ist essentiell in Ihrer
  Ausarbeitung,
  als auch später in Ihrer Bachelorarbeit.

  Literaturverweise werden in \LaTeX{} mittels
  \texttt{\textbackslash cite} getätigt.
  Verweise stehen hinter der Aussage, die sie stützen sollen,
  oder hinter dem Namen eines referenzierten Konzepts oder Eigenschaft
  und gehören in den Satz hinein.
  Beachten Sie folgendes Beispiel:
  \begin{itemize}
    \item In ``Artificial Intelligence: A Modern Approach''~\cite{russell2002artificial}
      werden moderne Ansätze und Techniken
      der künstlichen Intelligenz vorgestellt.
    \item bei direkten Zitationen wir zusätzlich die Seite angegeben wo das Zitat zu finden ist, z.B. ``...''~\cite[p.1]{russell2002artificial}
  \end{itemize}

  Am Ende der Ausarbeitung werden die referenzierten Quellen automatisch
  aufgelistet.

  \subsection{Bibtex-Einträge und \texttt{references.bib}}

  Um mit \texttt{\textbackslash cite} referenzieren zu können,
  müssen die entsprechenden Einträge in der Datei \texttt{references.bib}
  stehen.
  Sie finden in der Vorlage bereits den beispielhaften Eintrag für
  ``Artificial Intelligence: A Modern Approach''~\cite{russell2002artificial}.

  Für Literaturquellen empfiehlt es sich in aller Regel
  \url{https://scholar.google.com} zu nutzen.
  Hier kann man sich direkt den Bibtex-Eintrag einer Quelle
  angeben lassen und muss diesen so nicht selbst zusammenstellen.

  Beachten Sie auch den Artikel
  ``Common Errors in Bibliographies''
  von John Owens.\footnote{%
    \url{https://www.ece.ucdavis.edu/~jowens/biberrors.html}}
  Des Weiteren ist der Vortrag von Simon Peyton Jones \href{https://www.microsoft.com/en-us/research/academic-program/write-great-research-paper/}{``How to write a great research paper''} zu empfehlen.

  \appendix
  \section{Beispielhafter Anhang}
  Der Anhang wird mit dem \texttt{\textbackslash{}appendix}-Befehl eingeleitet.
  Alle folgenden Sektionen werden als Anhang formatiert.

\backmatter
\listoffigures
\listoftables
\bibliography{references}
\bibliographystyle{plaindin}

\end{document}

% Quellen
%
% Why You’ll Probably Never Use Haskell in Production
% https://henrydangprg.com/2019/05/27/why-youll-probably-never-use-haskell-in-production/
%
% Choosing Haskell isn’t a stand-in for good software design
% https://ozataman.medium.com/choosing-haskell-isnt-a-stand-in-for-good-software-design-7d893882f963
%
% A History of Haskell: Being Lazy With Class
% https://dl.acm.org/doi/10.1145/1238844.1238856
